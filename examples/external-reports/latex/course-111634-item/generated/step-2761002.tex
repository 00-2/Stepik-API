
\newpage
\stepurl{https://stepik.org/lesson/667378/step/4}
\stepstatistics{2761002}{0.68}{0.95}{0.68}{38}{code}

\begin{question}
\hypertarget{ux43fux435ux440ux435ux445ux432ux430ux442-ux432ux440ux430ux436ux435ux441ux43aux438ux445-ux43fux43bux430ux43dux43eux432}{%

\subsection{Перехват вражеских

планов}\label{ux43fux435ux440ux435ux445ux432ux430ux442-ux432ux440ux430ux436ux435ux441ux43aux438ux445-ux43fux43bux430ux43dux43eux432}}



Выполняя задание Центра, Штирлиц остался один в кабинете Мюллера. В

бумагах группенфюрера разведчик обнаружил описание шифровки,

используемой фашистами для описания перемещений своих войск. Они шифруют

цифры, пробелы, точки и дефисы в виде двухэтажных сообщений, в которых

каждый символ занимает две соседние строки. В ширину каждый символ может

кодироваться количеством символов от 1 до 3. Таблица, которую нашёл

Штирлиц, показана ниже.



\begin{longtable}[]{@{}

  >{\raggedright\arraybackslash}p{(\columnwidth - 4\tabcolsep) * \real{0.3333}}

  >{\centering\arraybackslash}p{(\columnwidth - 4\tabcolsep) * \real{0.3333}}

  >{\raggedright\arraybackslash}p{(\columnwidth - 4\tabcolsep) * \real{0.3333}}@{}}

\toprule

\begin{minipage}[b]{\linewidth}\raggedright

Шифр

\end{minipage} & \begin{minipage}[b]{\linewidth}\centering

Символ

\end{minipage} & \begin{minipage}[b]{\linewidth}\raggedright

Примечание

\end{minipage} \\

\midrule

\endhead

\texttt{\textbar{}\ \ \ \ \ \ \ \ \ \ \ \ \ \textbar{}} & 1 & ~ \\

\texttt{\^{}\textbar{}\ \ \ \ \ \ \ \ \ \ \ \ \ /\_} & 2 & ~ \\

\texttt{\_)\ \ \ \ \ \ \ \ \ \ \ \ \ \_)} & 3 & ~ \\

\texttt{\textbar{}\_\textbar{}\ \ \ \ \ \ \ \ \ \ \ \ \ ~\ \textbar{}} &

4 & Шифр в 3 символа в ширину \\

\texttt{\textbar{}-\ \ \ \ \ \ \ \ \ \ \ \ \ \_)} & 5 & ~ \\

\texttt{/\ \ \ \ \ \ \ \ \ \ \ \ \ ()} & 6 & ~ \\

\texttt{\textasciitilde{}/\ \ \ \ \ \ \ \ \ \ \ \ \ /} & 7 & ~ \\

\texttt{()\ \ \ \ \ \ \ \ \ \ \ \ \ ()} & 8 & ~ \\

\texttt{()\ \ \ \ \ \ \ \ \ \ \ \ \ ~/} & 9 & ~ \\

\texttt{/\textbackslash{}\ \ \ \ \ \ \ \ \ \ \ \ \ \textbackslash{}/} &

0 & ~ \\

\begin{minipage}[t]{\linewidth}\raggedright

\hfill\break

\texttt{.}\strut

\end{minipage} & . & Вверху пробел, внизу точка \\

\begin{minipage}[t]{\linewidth}\raggedright

\texttt{\_}\strut \\

\hspace*{0.333em}\strut

\end{minipage} & - & Вверху знак подчёркивания, внизу пробел - кодирует

минус \\

\begin{minipage}[t]{\linewidth}\raggedright

\texttt{~}\strut \\

\hspace*{0.333em}\strut

\end{minipage} & ~ & Пробел сверху и снизу кодирует пробел \\

\bottomrule

\end{longtable}



~



~



~



~



~



~



~



~



~



~



~



~



~



~



~



~



~



~



~



~



~



~



~



~



~



Напишите программу, которая будет преобразовывать фашистский шифр в

обычные строки.



\textbf{Входные данные:}



Вводится две строки, в первой из которых хранится верхняя часть

зашифрованного сообщения, а во второй - нижняя часть. Если строки имеют

разную длину, то считается, что длина всей шифровки - это длина более

короткой строки, т.е. лишний суффикс одной из строк игнорируется. Длина

каждой из строк не превышает 10\textsuperscript{6}.



\textbf{Выходные данные:}



Расшифрованная строка. Если строку расшифровать не удалось, программа

должна вывести NICHT.


\end{question}
