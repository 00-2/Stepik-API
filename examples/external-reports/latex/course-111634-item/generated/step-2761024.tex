
\newpage
\stepurl{https://stepik.org/lesson/667378/step/5}
\stepstatistics{2761024}{0.85}{1.00}{0.73}{26}{code}

\begin{question}
\hypertarget{ux448ux438ux444ux440ux43eux432ux43aux438-ux438ux437-ux446ux435ux43dux442ux440ux430}{%

\subsection{Шифровки из

Центра}\label{ux448ux438ux444ux440ux43eux432ux43aux438-ux438ux437-ux446ux435ux43dux442ux440ux430}}



Оперативную информацию из Центра Штирлиц получает с помощью шифровок.

Принцип шифрования такой: с помощью последовательности цифр кодируются

фигуры - полилинии, эллипсы и точки. Если эти фигуры нарисовать, то

полученная картинка визуально отображает необходимую информацию. Чтобы

не рисовать всё вручную, напишите программу, которая будет

преобразовывать шифровку в html-файл с svg-изображением того, что

написано в шифровке.



Шифровка устроена следующим образом. В начале каждой фигуры идёт число,

обозначающее её тип



\begin{itemize}

\tightlist

\item

  0 - полигон. Следующее число - это количество вершин, дальше идет по

  два числа на каждую вершину - её координаты.

\item

  1 - полилиния. Устроена так же, как и полигон.

\item

  2 - эллипс. Следующие два числа - x и y координаты центра эллипса,

  следующие два числа - его радиусы по x и по y, следующее число равно

  либо 0, либо 1. 0 означает, что эллипс закрашенный, 1 что это просто

  контур

\item

  3 - конец шифровки

\end{itemize}



\textbf{Входные данные:}



На первой строке через пробел вводятся натуральное число n - количество

чисел в шифровке (0 \textless{} n ≤ 10\textsuperscript{5}) и натуральное

число w - толщина линий в изображении, которое требуется получить (от 1

до 5).



Далее вводится n натуральных чисел от 0 до 1000 - сама шифровка. На всех

строках, кроме самой последней, вводится по 10 чисел, последняя строка

может содержать меньше, если n~ не делится на 10.



\textbf{Выходные данные:}



На первой строке нужно вывести \textless svg width="250"

~height="150"\textgreater. На место значение атрибута width нужно

поставить x координату правой границы самой правой фигуры, увеличенную

на w, на место атрибута height нужно поставить y координату нижней

границы самой нижней фигуры, увеличенную на w.



Далее каждую фигуру нужно выводить на отдельной строке. Формат каждой

фигуры:



\begin{itemize}

\tightlist

\item

  Полигон: \textless polygon stroke="black" stroke-width="2px"

  points="20,30 88,30 88,90"/\textgreater, где на место значения

  атрибута points нужно подставить координаты вершин полигона из

  шифровки. Обратите внимание на отсутствие пробелов между координатами.

\item

  Полилиния: \textless polyline stroke="black" stroke-width="2px"

  fill="none" points="92,30 92,90 160,30"/\textgreater, где на место

  значения атрибута points нужно подставить координаты точек полилинии

  из шифровки.

\item

  Эллипс: \textless ellipse stroke="black" stroke-width="2px"

  fill="white" cx="40" cy="50" rx="20" ry="40"/\textgreater, где на

  место значений атрибутов cx, cy, rx и ry нужно подставить координаты

  центра и радиусы эллипса из шифровки. Если в шифровке указано, что

  эллипс должен быть закрашенным, атрибут fill отсутствует.

\end{itemize}



На место значения атрибута stroke-width необходимо во всех фигурах

подставить значение w с суффиксом px.



На последней строке нужно вывести \textless/svg\textgreater{}



Отступы в начале строк нигде не делать.


\end{question}
