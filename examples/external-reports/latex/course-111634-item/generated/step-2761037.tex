
\newpage
\stepurl{https://stepik.org/lesson/667378/step/7}
\stepstatistics{2761037}{0.33}{0.50}{0.47}{9}{code}

\begin{question}
\hypertarget{ux441ux43eux43eux431ux449ux435ux43dux438ux435-ux434ux43bux44f-ux446ux435ux43dux442ux440ux430}{%

\subsection{Сообщение для

Центра}\label{ux441ux43eux43eux431ux449ux435ux43dux438ux435-ux434ux43bux44f-ux446ux435ux43dux442ux440ux430}}



У Штирлица всё практически готово для того, чтобы отправить в Центр

необходимые разведданные. Осталось лишь собрать все компоненты воедино:

расшифровать сведения о перемещениях фашистов, рассчитать, где они будут

находиться в указанный момент, и отправить в Центр координаты

противника, зашифрованные так, как указано в шифровке из Центра.



\textbf{Входные данные:}



Первые две строки - это зашифрованная фашистским шифром скорость

перемещения фашистов.



На следующих двух строках вводится так же зашифрованное число n -

количество точек в полилинии, по которой будут двигаться фашисты (0

\textless{} n ≤ 10\textsuperscript{3}).



На следующих n * 2 строках - зашифрованные координаты точек этих

полилиний через пробел (по два числа на каждой строке, но и на каждую

строку, в соответствии с фашистским шифром, две строки)



На следующей строке вводится интервал с момента выхода фашистских войск,

через который должны быть получены разведданные - вещественное число с

точностью до двух знаков после запятой.



При этом в ряде случаев фашистский шифр расшифровать не удастся - это

означает, что он содержит какую-то иную информацию.



\textbf{Выходные данные:}



Несколько строк должны содержать координаты фашистских войск через

указанный интервал на всём протяжении пути. Эти строки должны быть

зашифрованы способом из предыдущей задачи. Если шифр не удалось

расшифровать, программа должна вместо всего этого вывести слово Schwitz.


\end{question}
