
\newpage
\stepurl{https://stepik.org/lesson/667378/step/3}
\stepstatistics{2848633}{0.61}{0.91}{0.64}{57}{code}

\begin{question}
\hypertarget{ux441ux442ux440ux435ux43bux44cux431ux430-ux43fux43e-ux434ux432ux438ux436ux443ux449ux435ux43cux443ux441ux44f-ux43fux440ux43eux442ux438ux432ux43dux438ux43aux443}{%

\subsection{Стрельба по движущемуся

противнику}\label{ux441ux442ux440ux435ux43bux44cux431ux430-ux43fux43e-ux434ux432ux438ux436ux443ux449ux435ux43cux443ux441ux44f-ux43fux440ux43eux442ux438ux432ux43dux438ux43aux443}}



Штирлиц получил из Центра задание разведать планы фашистов по переброске

войск. Советская армия планирует в определённые моменты нанести по

фашистам артиллерийские удары, и Штирлиц должен предоставить информацию

о том, где будут в эти моменты времени находиться фашисты, чтобы

стрелять именно туда. Помогите Штирлицу выполнить свой долг, написав

программу, которая будет рассчитывать координаты местонахождения

фашистских войск.



\textbf{Входные данные:}



На первой строке вводится вещественное число v - скорость перемещения

фашистских войск в км/ч с точностью до 2 знаков после запятой (0

\textless{} v ≤ 10\textsuperscript{3}).



На второй строке вводится натуральное число n - количество точек,

образующих полилинию, по которой будут двигаться фашистские войска (0

\textless{} n ≤ 10\textsuperscript{3}).



Далее на n строках вводится по два числа - координаты каждой точки

полилинии движения войск в км. Каждая координата - это вещественное

число от -40000 до 40000 с точностью до двух знаков после запятой.



Далее вводится время, через сколько часов после начала маршрута движения

фашистских войск планируется артиллерийский удар.



\textbf{Выходные данные:}



На одной строке выводится два числа - координаты \textbf{x} и

\textbf{y}, где будут находиться фашистские войска в момент

артиллерийского удара с точностью до двух знаков после запятой. Если в

момент артиллерийского удара фашисты уже закончат свой маршрут,

программа должна вывести прочерк.


\end{question}
