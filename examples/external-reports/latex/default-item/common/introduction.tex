\chapter{Введение}
\section*{Назначение}
С помощью данного отчета можно оценить качество заданий \coursetexttointro на основе их  статистических характеристик,  выявить проблемные задания и внести в них изменения. В отчете выделены следующие характеристики  заданий: сложность задания, его дискриминативность и корреляция с общим баллом. Также проведен анализ опций для выявления нефункциональных дистракторов~--- неверных опций, на которые правильно отвечают почти все учащиеся.

\section*{Сложность задания}
Сложность задания определяется как процент учащихся, которые при первой попытке неправильно ответили на данное задание. Чем больше учащихся верно ответили на задание с первой попытки, тем менее трудным оно является. Оптимальная сложность задания~--- около 0.5, т.е. около 50\% учащихся верно ответили на задание с первой попытки.

\begin{flushleft}
\noindent\begin{tabularx}{\textwidth}{c@{\hspace*{5em}}X}
\hline
Сложность задания & \multicolumn{1}{c}{Интерпретация}\\
\hline
0.75--1.00 & Сложное задание\\
0.25--0.75 & Среднее по сложности\\
0.00--0.25 & Легкое задание\\
\hline
\end{tabularx}
\end{flushleft}

\section*{Дискриминативность задания}
Дискриминативность (или различительная способность) задания показывает, насколько сильно различаются сильно подготовленные и слабо подготовленные учащиеся. В данном отчете дискриминативность вычисляется с применением метода крайних групп, т.е. при расчете учитываются результаты учащихся, наиболее и наименее успешно справившихся со всеми тестовыми заданиями (в каждой из этих групп~--- по 30\% учащихся).

Низкая дискриминативность означает, что данное задание одинаково\linebreak успешно выполняют как сильно подготовленные учащиеся, так и слабо подготовленные. Типичными недостатками  задач с низкой дискриминативностью являются:
\begin{itemize}
\item излишняя сложная и запутанная формулировка задания;
\item неоднозначная формулировка задания;
\item очевидное решение задания;
\item зависимость результата от памяти или других индивидуальных особенностей учащегося, а не от уровня развития тех умений и навыков, для оценки которых разрабатывалось задание;
\item абсурдность некоторых опций;
\item существование опций, отмеченные как неправильные, но которые могут быть верными при некоторых обстоятельствах.
\end{itemize}

Если дискриминативность отрицательна, то это означает, что сильно подготовленные учащиеся отвечают на такое задание хуже, чем слабо подготовленные. Такие задания требуют более тщательной проверки: возможно, что в таких заданиях верная опция  была отмечена как неверная (или наоборот). 


\begin{flushleft}
\noindent\begin{tabularx}{\textwidth}{c@{\hspace*{2em}}X}
\hline
Дискриминативность & Интерпретация\\ 
\hline
0.30--1.00 & Высокая дискриминативность\\
0.10--0.30 & Средняя дискриминативность\\
0.00--0.10 & Низкая дискриминативность. Задание рекомендуется заменить\\
меньше 0.00& Отрицательная дискриминативность. Задание требует проверки или исключения\\
\hline
\end{tabularx}
\end{flushleft}



\section*{Корреляция с общим баллом}
Корреляция с общим баллом может принимать значения от -1 до 1 и показывает, насколько задание согласуется с остальными. Низкое значение корреляции (меньше 0.2) часто свидетельствует о том, что данное задание измеряет другую характеристику, нежели остальные задания.

Если корреляция с общим баллом отрицательная, то это может также свидетельствовать о низкой дискриминативности задания.


\section*{Анализ дистракторов}
Дистрактором называется неправильная, но правдоподобная опция в заданиях с выбором одного или нескольких правильных вариантов. При анализе дистракторов следует обращать внимание на те из них, которые были выбраны менее чем 5\% учащихся, так называемые нефункциональные дистракторы, так как при удалении их из задания, его характеристики практически не меняются.

Общие рекомендации при написании опций (в том числе, и дистракторов) таковы:
\begin{itemize}
\item верная опция не должна быть самой длинной или самой подробной,
\item необходимо избегать <<частично правильных опций>>, т.е. таких опций, которые могут оказаться верными в некоторых ситуациях,
\item в качестве дистракторов рекомендуется использовать типичные ошибки,
\item не должно быть грамматических подсказок, например, согласование в роде и/или числе, между заданием и опциями,
\item опции <<все из выше перечисленного>> и <<ничего из выше перечисленного>> не рекомендуются.
\end{itemize}


\chapter{Рекомендации по составлению заданий}

При составлении текста задания общие рекоммендации следующие:
\begin{itemize}
\item Задания должны быть связаны с учебными целями, т.е. со знаниями и навыками, которыми должен обладать учащийся после окончания курса.
\item Текст задания должен исключать всякую двусмысленность и неясность формулировок.
\item Текст задания формулируется предельно кратко, но без ущерба для понимания. 
\item Используемая в заданиях терминология не должна выходить за рамки курса.
\item Если задание содержит отрицание, то частицу НЕ или слово, выражающее отрицание, необходимо выделить (например, заглавными буквами или жирным текстом с подчеркиванием).
\item Текст задания должен исключать сложные синтаксические обороты, в том числе, двойное отрицание.
\item В тексте задании не используются слова, которые могут вызвать различное понимание учащихся, а также слова <<иногда>>, <<часто>>, <<всегда>>, <<все>>, <<никогда>>.
\end{itemize}

При составлении текста ответа (опций) следует руководствоваться следующими рекомендациями:
\begin{itemize}
\item По возможности, не следует делать текст ответов длинным.
\item Из ответов, как правило, исключаются все повторяющиеся слова путем ввода их в основной текст задания.
\item Все ответы должны быть похожими как по внешнему виду, так и по грамматической структуре.
\item Все ответы должны быть грамматически согласованными с основной частью задания.
\item Ответы должны быть примерно равны по длине: не следует формулировать правильный ответ заметно длиннее или короче, чем неправильные.
\item Одно и то же слово (или словосочетание, или однокоренное слово) не должно находиться и в тексте задания, и в правильном ответе.
\item Каждый неправильный ответ должен быть правдоподобным, внушающим доверие и убедительным для учащихся.
\item В неправильных ответах не следует использовать слова или термины, которые учащийся не может или не должен знать, в том числе, не относящиеся к содержанию курса.
\item Исключаются ответы, вытекающие один из другого или дополняющие друг друга.
\item При формулировке ответов не следует использовать выражения <<все перечисленные>>, <<ни один из перечисленных>>, <<все, кроме...>> или их аналоги.
\item Если ответы выражены в виде чисел, то рекомендуется располагать эти числа от меньшего к большему или наоборот.
\item В задании на выбор нескольких правильных ответов не должны быть все ответы правильные или только один правильный (учащиеся ожидают два и более правильных ответа, но не все из них).
\end{itemize}

